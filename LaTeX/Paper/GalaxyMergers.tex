% by Sean McKenna

% !TEX TS-program = pdflatex
% !TEX encoding = UTF-8 Unicode

% This is a simple template for a LaTeX document using the "article" class.
% See "book", "report", "letter" for other types of document.
\documentclass[11pt]{article} % use larger type; default would be 10pt
\usepackage[utf8]{inputenc} % set input encoding (not needed with XeLaTeX)

%%% PAGE DIMENSIONS
\usepackage{geometry} % to change the page dimensions
\geometry{a4paper} % or letterpaper (US) or a5paper or....
% \geometry{margins=2in} % for example, change the margins to 2 inches all round
% \geometry{landscape} % set up the page for landscape
%   read geometry.pdf for detailed page layout information

%\usepackage{graphicx} % support the \includegraphics command and options

% \usepackage[parfill]{parskip} % Activate to begin paragraphs with an empty line rather than an indent

%%% PACKAGES
\usepackage{booktabs} % for much better looking tables
\usepackage{array} % for better arrays (eg matrices) in maths
%\usepackage{paralist} % very flexible & customisable lists (eg. enumerate/itemize, etc.)
\usepackage{verbatim} % adds environment for commenting out blocks of text & for better verbatim
\usepackage{subfig} % make it possible to include more than one captioned figure/table in a single float
\usepackage{amssymb,amsmath} % for using AMS math packages
\usepackage{graphicx} % for displaying images in the paper
% These packages are all incorporated in the memoir class to one degree or another...

%%% HEADERS & FOOTERS
\usepackage{fancyhdr} % This should be set AFTER setting up the page geometry
\pagestyle{fancy} % options: empty , plain , fancy
\renewcommand{\headrulewidth}{0pt} % customise the layout...
\lhead{}\chead{}\rhead{}
\lfoot{}\cfoot{\thepage}\rfoot{}

%%% SECTION TITLE APPEARANCE
%\usepackage{sectsty}
%\allsectionsfont{\sffamily\mdseries\upshape} % (See the fntguide.pdf for font help)
% (This matches ConTeXt defaults)

%%% ToC (table of contents) APPEARANCE
%\usepackage[nottoc,notlof,notlot]{tocbibind} % Put the bibliography in the ToC
%\usepackage[titles,subfigure]{tocloft} % Alter the style of the Table of Contents
%\renewcommand{\cftsecfont}{\rmfamily\mdseries\upshape}
%\renewcommand{\cftsecpagefont}{\rmfamily\mdseries\upshape} % No bold!

%%% END Article customizations
\linespread{2}
%%% The "real" document content comes below...

\title{Galaxy Mergers and Star Formation using TreeSPH}
\author{Sean McKenna}
%\date{} % Activate to display a given date or no date (if empty),
         % otherwise the current date is printed 

\begin{document}
\maketitle


\section{Introduction}
	In about three billion years, scientists hypothesize that the Milky Way Galaxy will collide with the Andromeda Galaxy (M31) \cite{andromeda}. Currently, the Andromeda Galaxy is over $2.5x10^6$ ly away, and the two galaxies are approaching eachother at speeds around 120 km/s. It is important to note that this collision will occur about two billion years before the Sun nears the end of its lifetime and turns into a red giant; by this time, there is no doubt that the Sun will end all life on Earth due to intolerable temperatures \cite{andromeda}. Nevertheless, even if the galaxies do end up missing one another, the gravitational forces from their closer interactions will cause a smaller, faster orbit and the eventual collision and merging of the two galaxies.

	So what will happen to the Earth? There are two main predictions. One is that the solar system will be flung outside of the Milky Way Galaxy due to gravitational interactions with the larger Andromeda galaxy. This would result in a much darker night sky, with very few stars being anywhere near the system. Alternatively, it is also quite possible that the interactions will cause Earth to be near a region of starbursts \cite{site}, where the night sky will be lit up about once per year from the increasing rate of stars going supernova, as opposed to the current observed rate of two supernovae per century \cite{andromeda}.

	Ever since scientists have discovered that these two galaxies are on a ``collision course," researchers have wondered what happens during and after galactic interactions. Since the overall rate of these interactions has decreased with the expansion of the Universe after the Big Bang \cite{andromeda}, observable phenomena is less likely to be a useful tool for astrophysicists to discover how these interactions will affect the star systems both inside and outside the galaxies. To further complications, the timescale for galactic interactions is so huge that observations provide such a little snapshot in time. One of the key questions in research today is how star formation will be affected by galaxies interacting. To address this, computational astrophysicists have developed efficient algorithms for modeling galactic interactions which enable researchers to find an increase in the rate of star formation during these encounters \cite{mergers,mergerstats,andromeda,starburst,superwinds}.


\section{Concepts}
	Before delving into the details of any model, a basic understanding of cosmology and numerics is essential. For example, many models in astrophysics rely on numerical simulations to be studied. In many cases, semi-analytic techniques are used to approximate and simplify a problem down to one with a better physical understanding and simpler solutions (or integrals). This is the case of many dynamic systems and modeling in general. When creating simulations, there is always a delicate balance between accuracy and computational time, like in the simulations of merging galaxies done by Chilingarian and Di Matteo through the GalMer project \cite{GalMer}. Using algorithms with a decent amount of accuracy but yet also not over-calculating the accuracy is tricky. There are also a number of mathematical tricks that can be utilized to speed up the calculations but at the cost of some accuracy in the resulting data.

	Next, it is important to know what a galaxy actually is. In its most basic form, a galaxy is a collection of stars (and dust and gas) connected through their gravitational interactions \cite{galaxy}. In the same way planets orbit stars, stars can orbit galaxies. What they exactly orbit is less clear; the current hypothesis is that a black hole is at the center of most galaxies \cite{galaxy}. Along with this, there can be different formations of stars around a galaxy; this results in different kinds of galaxies such as: cluster, dwarf, elliptical, irregular, and spiral. Note that not just stars make up the galaxy; gas constitutes around ten percent of many galaxies \cite{starburst}. This gas is vital to stellar formation.

	To model a galaxy, knowledge of dark matter is required to fit the current understanding of how stars and dark matter interact through gravity. Dark matter is not something that is a straight-forward concept, nor is it something that is really well understood \cite{modelsintime}. An astronomer known as Zwicky discovered anomalies and missing matter in gravitational shifts of object but no other reason for the shift could be established. In theory, this gravitational shifting is a result of dark matter interacting with stars \cite{modelsintime}. Thus, dark matter is matter that causes gravitational interactions with matter around it but not any other kind of interaction, like electromagnetic interactions. In time, an average density of dark matter was calculated based on these distortions, and this value for dark matter is used in numerical simulations of galaxies interacting.

	Stars form from dense clouds of gas, but this phenomenon can also occur at faster rates inside of some galaxies. Commonly known as a starburst galaxy, there is an abnormally high rate of star formation in these regions of space. Galaxies of increased stellar formation are also just called starbursts. Related with that is the concept of superwinds which can transfer gas and also cause these starbursts \cite{superwinds}. There are some complications with detecting some starburst galaxies. Some of these problems arise from dense clouds in galaxies of higher redshift which cause the star formation to be more hidden from detection on Earth \cite{starburst}.  Star formation rate (SFR) is a concept for quantifying stellar formation in galaxies, based on how quickly stars are forming. Another concept is the star formation efficiency (SFE) \cite{mergerstats}, which is explained by Equation \ref{sfe}.
  \begin{equation}
    \label{sfe}
    SFE(t) = \frac{M_{gas \to \star}(t)}{M_{gas}(t)}
  \end{equation}
This equation is, at some time $t$, the ratio of the amount of gas turned into stars, compared with the amount of gas remaining in the galaxy, effectively measuring how efficiently stars are being formed. Rates of star formation and starbursts are important because they enable scientists to understand more about the history and the future of the Universe based on simulations that comply with observations.

	When galaxies interact, each star has many gravitational forces being exerted upon it. This means that gravity of some mass (an entire galaxy) interacts with that of another galaxy. The masses can be about the same or really different. But what happens during these interactions? Many times energy is redistributed and gases or elements are shifted towards higher density areas. This results in areas of more densely-packed gas and thus an increased formation rate of stars. Also, mixing the gases of supernova remnants and other gases or winds in the interstellar medium can often result in a starburst as well. Numerical simulations become an essential step in modeling galactic interactions because every star in a galaxy causes a gravitational force on each other star in the galaxy \cite{mergers,mergerstats,LBG,superwinds}. These interactions result in billions of calculations per snapshot, which is why algorithms have been formulated to speed up this process. For this process, more will be discussed on how one particular algorithm works.

	Besides just interactions, there is a type of galaxy interactions known as mergers. Another common name is a galaxy collision. However, unlike real collisions, it is extremely unlikely that any stars will collide because of the vast amount of space in-between all the stars in a galaxy \cite{andromeda}. An example of a merger is NGC 4676, also known as ``The Mice." Its main feature is the visible tail of the two galaxies as they slowly merge into one. Furthermore, unlike a collision, mergers are re-birth of a new type of galaxy, usually an elliptical galaxy which results from two spiral galaxies merging \cite{andromeda}. Like in galactic interactions, the gravitational tidal forces of the merge cause nebulae and superwinds to be disturbed which results in gas collapsing into stars \cite{knol}.


\section{Model}
	To simulate a system, a semi-analytic model must first be established. This is essential to simplifying the equations and the system. Plus, with many simulations, a statistical approach to modeling mergers can still provide fairly accurate results \cite{mergerstats}. This makes sense since, even with inaccuracies, a large amount of simulations can average out these inaccuracies and get the general picture of the model. A model was established by Di Matteo and Combes in 2008 at an institute in France, and they continued their work with collaborator Chilingarian from Moscow with the GalMer project \cite{mergers,mergerstats}. The models described in the two different simulation scenarios incorporate a set of assumptions, constants, and parameters. Assumptions are simplifications that affect the accuracy of the solution of the problem, but these are carefully taken into consideration to minimize the error. Constants are accepted values of physical phenomena. Lastly, the parameters define the simulations such that they reflect real, observable phenomenon in the Universe. As such, these simulations are calibrated to match real data before being analyzed.

	For the two studies, the galaxies were modeled as a ``spherical non-rotating dark-matter halo" \cite{mergers}. The specifics are difficult to explain, but this just means that stars, dark matter, and gas are the main constituents of the galaxy. The galaxies are also considered to be Plummer spheres, or a spherical three-dimensional object with a known density gradient. The first study assumed a mass ratio of the two galaxies as 1:1 \cite{mergerstats}, while the second study took a set of different mass ratios from 1:1 to 1:10 \cite{mergers}. Also, the second study varied the galaxy types, ages, angles of collision, and spins of galaxies colliding.

	The remaining parameters of the model help define the equations for gravitational interactions. For the two models, the amount of stars and dark matter is represented by $N = 120,000$ \cite{mergers,mergerstats}. Along with that, a set of constant values were parameterized based on observations of mergers. The temperature is set to $10^4$ K for each case; the gas is assumed to be isothermal.

	Variables for a simulation are essential; they denote what is changing with every simulation. Analyzing changes in the simulations result in verification of theories or formulations of new hypotheses. For the first study, differences in the distance and spin orientation between galaxies are simulated. Spins for a set of galaxies can be parallel in their angular motion or anti-parallel. These are known as direct and retrograde encounters, respectively. The second study encompasses a larger project, the GalMer project \cite{GalMer}, which incorporates a set of varying interactions based on galaxy type, angle of interaction, mass ratios, and spin orientation. The angle of interaction is simply the angle that the two galaxies form as they merge.

	With the model established, a computational physicist must know how to simulate it. Galaxy interactions are commonly simulated with the TreeSPH algorithm \cite{mergers,mergerstats,treesph,algorithm,LBG}. This process computes the gravitational tidal forces between all the different $N$ stars in one galaxy with the $N$ stars in another galaxy \cite{LBG}. Figuring out the gravitational interactions and overall hydrodynamics of the system result in a simulation that occurs over a certain timeframe. Most numerical simulations are snapshots in time that can be pieced together to form an animation.


\section{Numerical Simulations}
	TreeSPH is an algorithm (used in galactic simulations) for calculating gravitational interactions. An algorithm is simply a procedural, step-by-step method of solving a problem, like how to solve a maze. There are two main components of the algorithm that cause it to be rather efficient in running time and flexibility. Each component is also Lagrangian, so the solutions are a result of summing up many little simplified pieces of the problem \cite{treesph}. This makes the algorithm susceptible to many different boundary conditions. Altogether, this makes the algorithm very flexible and decently fast considering the medium resolution achieved in the simulations.

	\begin{figure}
	  \begin{center}
	    \includegraphics[width=275px]{box.jpg}
	    \caption{Visual of the box-cutting method, repeatedly slicing volumes into 8 pieces \cite{algorithm}.}
	  \end{center}
	\end{figure}

	The first component of the algorithm is the hierarchical tree method. This is the longest step of the algorithm \cite{treesph}, since all the gravitational forces must be calculated between all the stars. To create the algorithm, Barnes and Hut stored the tidal forces in a tree-like data structure, breaking down the volumes of space into volume containing a single star \cite{algorithm}. In each step, the current piece of volume is cut into eight pieces and empty pieces are discarded, see Figure 1. It continues recursively until the current volume piece contains a single star. This volume cutting process generates the tree. From the tree, sets of particles (stars and dark matter) are clustered together to obtain a net gravitational potential for each cluster which is then applied to a single star's interaction calculations, resulting in less computation for each star. This process does not significantly alter the accuracy of the simulation as well \cite{algorithm}.

	The second component of the algorithm is the process of smoothed particle hydrodynamics (SPH). This is the second-longest computational step of the algorithm \cite{treesph}. Just having gravitational forces of particles does not necessarily illustrate how these particles will interact with the gas as a whole. Modeling the galaxies as particles of a fluid, a number of assumptions are taken into consideration, one of which is ignoring mass conservation and finding a gradient of the physical field of the fluid \cite{treesph}. Interestingly, the lack of mass conservation results in a cancellation of the momentum due to symmetry so that momentum is still conserved, despite the assumptions \cite{treesph}. This is one of the reasons this component still gives decently accurate results. Also, another algorithm (or trick) is employed in this process when analyzing the equations of state for a fluid. This step is where the assumption of the system being isothermal is essential, and the required method is called leap-frog integration \cite{treesph}. This method is simply a way to calculate the position and velocity of particles at certain intervals of time, using the overall pattern to simulate the system. This trick enables individual particles to have their own time intervals, allowing for more or less resolution for certain particles; this is not a feature allowed by most algorithms. The overall motion of the particles in the gas are then smoothed together with the gravitational interactions to successfully simulate the merging galaxies.

	Any algorithm can be analyzed by its computational complexity, often referred to as the Big-Oh function or the running time complexity of an algorithm. This function is a rough estimate of the running time for an algorithm based only on the number $N$ in the system and basic mathematical functions. Modeling gravitational interactions between $N$ stars is known as an $N$-body problem, a computationally slow problem unless a special algorithm is employed \cite{algorithm}. For every $N$ star, the interactions between the other $N$ stars must be calculated. The resulting number of calculations is then $O(N^2)$. This is the heart of the $N$-body problem, and this is why many scientists look for shortcuts to reduce this calculation. If the algorithm takes polynomial time to get a solution, it will take increasingly more time on larger systems. The use of the TreeSPH algorithm reduces this complexity to $O(N\text{ log}N)$ which scales at a much slower rate for large $N$ than the $N$-body algorithm. Again, some resolution and accuracy is lost in this algorithm, but the benefits of faster computation outweigh the accuracy since there are already many assumptions taken into account \cite{algorithm}.

	Aside from the TreeSPH algorithm, there are faster algorithms that can be up to ten times faster \cite{improvedalgorithm}. In this paper, the authors altered the TreeSPH code to cluster together cells of stars and compare the gravitational interactions against other clustered cells of stars. This results in even more approximations and loss of accuracy, but it dramatically decreases the computational time. The running time complexity is $O(N)$, the ideal linear running time sought by most algorithms. It is an improvement over the fast multipole method (FMM) since it still uses the tree, as in TreeSPH. Unfortunately, faster usually means less resolution in the resulting simulation.

\begin{figure}
  \begin{center}
    \includegraphics[width=400px]{merger.jpg}
    \caption{Merger simulation of two galaxies on a retrograde encounter with their gas, stars, and dark matter from top to bottom. Each frame is 50 kpc by 50 kpc \cite{mergerstats}.}
  \end{center}
\end{figure}

\section{Results}
	An example of a simulated merger can be seen in Figure 2. Within the first study of same-type giant galaxy mergers, a general increase in the SFR is found for merging galaxies than ones that just interacted but did not merge (``fly-bys") \cite{mergerstats}. The researchers claim that mergers do not always result in starbursts and that not all galaxy interactions will create new stars. They argued that their statistics for 240 total interactions showed a general increase in the rates of star formation; even though, there were cases where no increase was found. Another interesting discovery was that the higher the maximum SFR, then the faster that rate would go back down, since the increased tidal forces resulted in less interaction time. Lastly, they found that retrograde (anti-parallel spin) mergers resulted in an overall higher SFR. They hypothesized that decreased gravitational tidal interactions leave more dense regions of space, which are more likely to starburst.

	The second study incorporated a larger amount of variables but also only presented results off of a subset of those variables. The variables of focus were the different kinds of galaxies interacting and the angular momentum or spin of these mergers. Like in the previous study, there was found an increase in the SFR for retrograde encounters \cite{mergers}. Then, agreeing with the results of the retrograde encounters and the previous study, a negative correlation was established between the gravitational forces and the strength of the amplitude of the starburst (related to the peak SFR). They also found that the amount of gas in the galaxy to start with was not a main parameter for the SFE. Note that this is the efficiency of the formation, not just the rate, since the rate of stellar formation does increase with the amount of gas in a galaxy \cite{LBG}. With the simulations, about $15\%$ of the interactions had a starburst with a SFR increase fivefold or more, which shows how rare these starbursts can be \cite{mergers}. This states that all interactions will not necessarily result in a starburst. They also found that the process of merging lasts, on average, about $10^8$ years, which is a rather small amount of time on a galactic scale.

	Many of the results of the second study were not completed since the project is on-going with plans for future papers. The project, known as GalMer \cite{GalMer}, encompasses all of the different merger interactions as laid out in their model. All of the data is available, frame-by-frame, on their website. For their study, a little over a thousand simulations were analyzed and stored on their website, for a total of 0.9 terabytes. To break down each interaction, it consists of 50-70 snapshots with a difference in each of fifty million years and each snapshot around 12 megabytes \cite{mergers}. This results in each simulation to be almost one gigabyte large. The whole purpose of the project is making the simulations and data available to other scientists. With virtual observatory (VO) tools, scientists can easily get metadata into applications that use standard formats for stellar data. Some of these tools include: FITS tables, IFU spectra, color-magnitude plots, or FITS images \cite{mergers}. Also, some of the merging galaxy data that is available is the velocity dispersion, gas density, stellar density, and stellar metallicity. With this data and tools for analysis, the virtual observatory can be used to compare observational data with the simulations very easily and efficiently.

	Both of the studies effectively show that stellar formation rates are increased during interactions and mergers \cite{mergers,mergerstats}. However, it is equally important to note that the interactions will not necessarily result in an increase. Furthermore, the two studies distinguish the efficiency of the stellar formation, as opposed to the SFR, since efficiency measures ratios of mass for the star and the total gas \cite{mergerstats} as in Equation \ref{sfe}. SFE usually increases at a delay to the SFR yet also remains higher after initial increases. Also, the SFE was not found to increase with simply more gas in a galaxy, since the mass of the gas remaining is also taken into account. Future work will continue on the GalMer project to explore more types and variations of galactic interactions, including simulations of higher resolution and accuracy. All in all, these mergers were statistically shown to increase the stellar formation rate.


\section{Conclusion}
	Simulations show increased stellar formation rates for interacting and merging galaxies increases \cite{mergers,mergerstats,LBG}. These results were found through simulations generated from repeated use of the TreeSPH algorithm which employs a simplified analysis of the hydrodynamic and gravitational interactions of two galaxies. It is vital to know that work with galactic mergers is not complete; projects like GalMer are continuing the simulations to analyze more features of mergers. Another benefit of this project is the ability to compare the simulations with observations using built-in tools. So, with more observations on mergers over time, their rates and motion can be compared against the merger model. The field of computational astrophysics of interacting galactic systems has only begun to model behavior that increases understanding of the past, present, and future of the Universe.


\bibliographystyle{abbrv}
\bibliography{sources}


\end{document}
