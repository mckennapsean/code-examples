% by Sean McKenna

\documentclass[xcolor=dvipsnames]{beamer}
\title{Basic Guide to Linear Algebra}
\author{CSC321 Computer Graphics}
\date{Sean McKenna
\\ 04 April 2011}

% the best, most detailed, and cleanest theme for LaTeX Beamer, my fav!
\usetheme{Boadilla}

\usepackage{amsmath}
\usepackage{graphicx}
\usepackage{listings}
\lstset{language=java}

\begin{document}
  \maketitle

\begin{frame}
\frametitle{Addition of vectors}
  \begin{align*}
    \vec{u} & = (u_x, u_y, u_z) \\
    \vec{v} & = (v_x, v_y, v_z) \\
    \vec{u} + \vec{v} & = (u_x + v_x, u_y + v_y, u_z + v_z)
    \end{align*}
  \end{frame}

\begin{frame}
\frametitle{Addition of vectors}
\framesubtitle{Example}
  \begin{align*}
    \vec{u} & = (6, 5, 4) \\
    \vec{v} & = (1, 2, 3) \\
    \vec{u} + \vec{v} & = (7, 7, 7)
    \end{align*}
  \end{frame}

\begin{frame}
\frametitle{Subtraction of vectors}
  \begin{align*}
    \vec{u} & = (u_x, u_y, u_z) \\
    \vec{v} & = (v_x, v_y, v_z) \\
    \vec{u} - \vec{v} & = (u_x - v_x, u_y - v_y, u_z - v_z)
    \end{align*}
  \end{frame}

\begin{frame}
\frametitle{Subtraction of vectors}
\framesubtitle{Example}
  \begin{align*}
    \vec{u} & = (6, 5, 4) \\
    \vec{v} & = (1, 2, 3) \\
    \vec{u} - \vec{v} & = (5, 3, 1)
    \end{align*}
  \end{frame}

\begin{frame}
\frametitle{Dot product of vectors}
  \begin{align*}
    \vec{u} & = (u_x, u_y, u_z) \\
    \vec{v} & = (v_x, v_y, v_z) \\
    \vec{u} \cdot \vec{v} & = (u_x * v_x) + (u_y * v_y) + (u_z * v_z)
    \end{align*}
  \end{frame}

\begin{frame}
\frametitle{Dot product of vectors}
\framesubtitle{Example}
  \begin{align*}
    \vec{u} & = (6, 5, 4) \\
    \vec{v} & = (1, 2, 3) \\
    \vec{u} \cdot \vec{v} & = 6 + 10 + 12 = 28
    \end{align*}
  \end{frame}

\begin{frame}[fragile]
\frametitle{Dot product of vectors}
\framesubtitle{Example (continued)}

  $\vec{u}$ and $\vec{v}$ are $N$ dimensional vectors.

  Here they are represented by arrays.


\begin{lstlisting}

  double dotProduct = 0.0;
  for( i = 0; i < N; i++ ) {
    dotProduct += u[i] * v[i];
  } // for

\end{lstlisting}

\end{frame}

\begin{frame}
\frametitle{Dot product of vectors}
\framesubtitle{Properties}

  $\vec{u} \cdot \vec{v} = |\vec{u}| * |\vec{v}| * cos(\psi)$, where $\psi$ is the angle between $\vec{u}$ and $\vec{v}$

  \end{frame}

\begin{frame}
\frametitle{Magnitude of a vector}
  \begin{align*}
    \vec{u} & = (u_x, u_y, u_z) \\
    | \vec{u} | & = \sqrt{(u_x)^2 + (u_y)^2 + (u_z)^2}
    \end{align*}
  \end{frame}

\begin{frame}
\frametitle{Magnitude of a vector}
\framesubtitle{Example}
  \begin{align*}
    \vec{u} & = (2, 3, 4) \\
    | \vec{u} | & = \sqrt{2^2 + 3^2 + 4^2} = \sqrt{29} \approx 5.385
    \end{align*}
  \end{frame}

\begin{frame}[fragile]
\frametitle{Magnitude of a vector}
\framesubtitle{Example (continued)}

  $\vec{u}$ is an $N$ dimensional vector.

  Here it is represented by arrays.

  \begin{lstlisting}{}
  double dotProduct = 0.0;
  for( i = 0; i < N; i++ ) {
    dotProduct += u[i] * u[i];
  } // for
  double magnitude = Math.sqrt( dotProduct );
    \end{lstlisting}

\end{frame}

\begin{frame}
\frametitle{Magnitude of a vector}
\framesubtitle{Properties}

  The $|\vec{u}|$ is equal to the square root of the dot product of $\vec{u}$ and itself.

  \end{frame}

\begin{frame}
\frametitle{Scalar multiplication of a vector}

  Here $s$ is a scalar---that is, $s$ is a number (NOT a vector).

  \begin{align*}
    \vec{u} & = (u_x, u_y, u_z) \\
    s * \vec{u} & = (s*u_x, s*u_y, s*u_z) 
    \end{align*}
  \end{frame}

\begin{frame}
\frametitle{Scalar multiplication of a vector}
\framesubtitle{Example}

  \begin{align*}
    s & = 2 \\
    \vec{u} & = (1, 2, 3) \\
    s * \vec{u} & = (2, 4, 6) 
    \end{align*}
  \end{frame}

\begin{frame}
\frametitle{Normalization of a vector}

  Normalization of a vector $\vec{u}$ produces
  a new vector $\hat{n}$ that points in the same
  direction as $\vec{u}$ but has a length (magnitude)
  equal to one.

  \begin{align*}
    \hat{n} & = \frac{1}{|\vec{u}|} \vec{u} \\
    \end{align*}

  \end{frame}

\begin{frame}
\frametitle{Cross product of vectors}
  \begin{align*}
    \vec{u} & = (u_x, u_y, u_z) \\
    \vec{v} & = (v_x, v_y, v_z) \\
    \vec{u} \times \vec{v} & = ((u_y * v_z) - (u_z * v_y), (u_z * v_x) - (u_x * v_z), (u_x * v_y) - (u_y * v_x))
    \end{align*}
  \end{frame}

\begin{frame}
\frametitle{Cross product of vectors}
\framesubtitle{Mnemonic}

  Let's define unit vectors that point in the positive
  direction along the $x$, $y$, and $z$ axes.
  Let's also define the two vectors that we want to 
  cross multiply.

  \begin{align*}
    \hat{i} & = (1, 0, 0) \\
    \hat{j} & = (0, 1, 0) \\
    \hat{k} & = (0, 0, 1) \\
    \vec{u} & = (u_x, u_y, u_z) \\
    \vec{v} & = (v_x, v_y, v_z) \\
    \end{align*}
  \end{frame}

\begin{frame}
\frametitle{Cross product of vectors}
\framesubtitle{Mnemonic (continued)}

  Can you describe a way to remember the
  formula for computing the cross product
  of two vectors?

  \begin{align*}
    \left[ \begin{array}{rrr}
      \mathbf{\color{green} \hat{i}} & \hat{j} & \hat{k} \\
      u_x & \mathbf{\color{green} u_y} & \mathbf{\color{red} u_z} \\
      v_x & \mathbf{\color{red} v_y} & \mathbf{\color{green} v_z} 
      \end{array} \right]
    \left[ \begin{array}{rrr}
      \mathbf{\color{red} \hat{i}} & \hat{j} & \hat{k} \\
      u_x & u_y & u_z \\
      v_x & v_y & v_z 
      \end{array} \right]
    \end{align*}
  \end{frame}

\begin{frame}[fragile]
\frametitle{Cross product of vectors}
\framesubtitle{Mnemonic (continued)}

  \begin{itemize}
    \item $\vec{w} = \vec{u} \times \vec{v}$
    \item Here we suppose that we have a class to model vectors
      and that the class provides getters and setters.
    \end{itemize}

\begin{lstlisting}
  w.setX( u.getY() * v.getZ() - 
          u.getZ() * v.getY() ) ;
  w.setY( u.getZ() * v.getX() - 
          u.getX() * v.getZ() ) ;
  w.setZ( u.getX() * v.getY() - 
          u.getY() * v.getX() ) ;
  \end{lstlisting}

\end{frame}

\begin{frame}
\frametitle{Cross product of vectors}
\framesubtitle{Mnemonic (continued)}

  \begin{itemize}
    \item You can use your right hand to remember the coordinate system (forefinger = x, middle finger = y, and thumb = z).
    \item How can I use my left hand to remember which are the
      positive directions on the coordinate axes in a left-handed
      coordinate system? (POV-Ray uses this coordinate system.)
    \end{itemize}

  \end{frame}

\begin{frame}
\frametitle{Cross product of vectors}
\framesubtitle{Properties}

  \begin{itemize}
    \item Let's anchor $\vec{u}$ and $\vec{v}$ at a common
      point $\mathbf{b}$.
    \item Let $\mathbf{a} = \mathbf{b} + \vec{u}$ and
      $\mathbf{c} = \mathbf{b} + \vec{v}$.
    \item That is to say, the two vectors point, respectively,
      from $\mathbf{b}$ to $\mathbf{a}$ and from $\mathbf{b}$
      to $\mathbf{c}$.
    \item The lines segments $\mathbf{ab}$ and $\mathbf{bc}$ are
      two sides of a parallelogram.
    \end{itemize}

  \end{frame}

\begin{frame}
\frametitle{Cross product of vectors}
\framesubtitle{Properties (continued)}

  \begin{itemize}
    \item The cross product of $\vec{u}$ and $\vec{v}$ points perpendicular to the plane that line segments $\mathbf{ab}$ and $\mathbf{bc}$ form.
    \item $\vec{u} \times \vec{v} = |\vec{u}| * |\vec{v}| * sin(\psi) * \mathbf{n}$, where $\psi$ is the angle between $\vec{u}$ and $\vec{v}$ and $\mathbf{n}$ is the perpendicular vector to the plane.
    \item Also, the magnitude of the cross-product is the area of the parallelogram formed by the two vectors.
    \end{itemize}

  \end{frame}

\begin{frame}
\frametitle{Cross product of vectors}
\framesubtitle{Properties (continued)}
  \begin{itemize}
  \item For any two distinct, non-zero vectors it is true that:
  \item $\vec{u} \times \vec{v} \neq \vec{v} \times \vec{u}$
  \item but $\vec{u} \times \vec{v} = - (\vec{v} \times \vec{u})$
  \end{itemize}
  \end{frame}

\begin{frame}
\frametitle{Applications of vector arithmetic}
\framesubtitle{Shading}

  \begin{itemize}  
    \item Let $\vec{N}$ be a unit vector that is normal to 
      a flat surface.
    \item Let $\vec{L}$ be a unit vector that points from
      the surface to a source of light.
    \item Suppose that the angle between $\vec{N}$ and $\vec{L}$
      is no greater than $\pi/2$ radians.
    \end{itemize}

  \end{frame}

\begin{frame}
\frametitle{Applications of vector arithmetic}
\framesubtitle{Shading (continued)}

  Define a function $f(\vec{N}, \vec{L})$ that
  will produce a value in $[0,1]$ that can be used to
  shade the surface.
  \begin{itemize}
    \item $f(\vec{N},\vec{L}) = 1$ when $\vec{N} \parallel \vec{L}$.
    \item $f(\vec{N},\vec{L}) = 0$ when $\vec{N} \perp \vec{L}$.
    \end{itemize}

  \end{frame}

\begin{frame}
\frametitle{Applications of vector arithmetic}
\framesubtitle{Shading (continued)}

  \begin{itemize}
    \item If the angle between $\vec{N}$ and
      $\vec{L}$ is greater than $\pi/2$, then we should be careful of getting negative values from the cosine function. The full expression for shading is below.

    \item $f(\vec{N},\vec{L}) = \vec{N} \cdot \vec{L}$
    \item UNLESS, $f(\vec{N},\vec{L}) < 1$ then set $f(\vec{N},\vec{L}) = 0$
  \end{itemize}

  \end{frame}

\begin{frame}
\frametitle{Applications of vector arithmetic}
\framesubtitle{Reflection}

  Let's consider the reflection of a ray of light
  from a surface.
  \begin{itemize}
    \item $\vec{I}$, $\vec{N}$, and $\vec{R}$ are unit vectors.
    \item $\vec{I}$ is the incident ray---it points from a 
      source of light to the surface.
    \item $\vec{N}$ is normal to the surface.
    \item $\vec{R}$ is the reflected ray---it points away from the
      surface.
    \item The law of reflection tells us that the angle between
      $\vec{I}$ and $\vec{N}$ equals the angle between
      $\vec{N}$ and $\vec{R}$.
    \end{itemize}

  \end{frame}

\begin{frame}
\frametitle{Applications of vector arithmetic}
\framesubtitle{Reflection (continued)}

  \begin{itemize}
  \item Define a function $\vec{f}(\vec{I},\vec{N})$ that returns $\vec{R}$.
  \item $\vec{R} = \vec{f}(\vec{I},\vec{N}) = \vec{N} \times (\vec{N} \times \vec{I})$
  \item Alternatively, one can invert the value of the component of $\vec{I}$ that is present in $\vec{N}$.
  \end{itemize}

  \end{frame}

\begin{frame}
\frametitle{Applications of vector arithmetic}
\framesubtitle{Computing normals}

  \begin{itemize}
    \item Let $\vec{a}$, $\vec{b}$, and $\vec{c}$ be three
      vectors that point from the origin to the vertices
      of the triangular face of a polyhedron.
      When the triangle is viewed from the outside
      of the polyhedron, $(\vec{a}, \vec{b}, \vec{c})$ 
      is a counterclockwise ordering of the vertices.

    \item Define a function $\vec{f}(\vec{a}, \vec{b}, \vec{c})$
      that returns a unit normal vector for the surface.

    \item $\vec{f}(\vec{a}, \vec{b}, \vec{c}) = (\vec{b}-\vec{a}) \times (\vec{c}-\vec{b})$
    \end{itemize}

  \end{frame}

\begin{frame}
\frametitle{Applications of vector arithmetic}
\framesubtitle{Interpolation}

  The following functions each return a vector.
  Those vectors, like the vector arguments of the
  functions, describe the location of points in space.

  \begin{align*}
    \vec{f}(t, \vec{a}, \vec{b}) & = (1 - t) \vec{u} + t \vec{v} \\
    \vec{f}(t, \vec{a}, \vec{b}, \vec{c}) & = 
      f( t, \vec{f}(t, \vec{a}, \vec{b}), \vec{f}(t, \vec{b}, \vec{c})) \\
    \vec{f}(t, \vec{a}, \vec{b}, \vec{c}, \vec{d}) & =
      \vec{f}(t, \vec{f}(t, \vec{a}, \vec{b}, \vec{c}),
           \vec{f}(t, \vec{b}, \vec{c}, \vec{d}))
    \end{align*}
  \end{frame}

\begin{frame}
\frametitle{Applications of vector arithmetic}
\framesubtitle{Interpolation (continued)}
  
  Let's be informal.
  We will allow ourselves to say  ``the point $\vec{a}\;\;$''
  even though we know that $\vec{a}$ is a vector and not a point.
  Maintaining a strict distinction between a point and the
  vector that extends from the origin to the point is too 
  cumbersome.

  \end{frame}

\begin{frame}
\frametitle{Applications of vector arithmetic}
\framesubtitle{Interpolation (continued)}

  Express $\vec{f}(t, \vec{a}, \vec{b}, \vec{c})$ as 
  a polynomial. Differentiate. Evaluate the derivative
  at $t = 0$ and $t = 1$.

  \begin{align*}
    \vec{f}(t, \vec{a}, \vec{b}, \vec{c}) & = 
      \vec{f}( t, 
        \vec{f}(t, \vec{a}, \vec{b}), 
        \vec{f}(t, \vec{b}, \vec{c})) \\
      & = (1 - t) [(1 - t) \vec{a} + t \vec{b}] + 
        t [(1 - t) \vec{b} + t \vec{c}] \\
      & = ?? \\
    \frac{\vec{f}}{dt}( t , \vec{a}, \vec{b}, \vec{c}) & = slope of curve \\
    \frac{\vec{f}}{dt}( 0 , \vec{a}, \vec{b}, \vec{c}) & = straight-line curve \\
    \frac{\vec{f}}{dt}( 1 , \vec{a}, \vec{b}, \vec{c}) & = how the derivative changes with time \\
    \end{align*}

  \end{frame}

\begin{frame}
\frametitle{Applications of vector arithmetic}
\framesubtitle{Interpolation (continued)}

  If $0 \leq t \leq 1$\ldots
  \begin{itemize}
    \item Where is $\vec{f}(t, \vec{a}, \vec{b})$ relative
      to the line segment defined by $\vec{a}$ and $\vec{b}$?
    \item Where is $\vec{f}(t, \vec{a}, \vec{b}, \vec{c})$ relative
      to the triangle defined by $\vec{a}$, $\vec{b}$, and $\vec{c}$?
    \item Where is $\vec{f}(t, \vec{a}, \vec{b}, \vec{c}, \vec{d})$ relative
      to the quadrilateral defined by 
      $\vec{a}$, $\vec{b}$, $\vec{c}$, and $\vec{d}$?
    \end{itemize}

  \end{frame}

\begin{frame}
\frametitle{Applications of vector arithmetic}
\framesubtitle{Interpolation (continued)}

  An algorithm for constructing a polyhedral approximation
  to a sphere repeatedly cuts lines segments at their midpoints.
  An algorithm for drawing the Koch Snowflake repeatedly
  cuts line segments in thirds.

  \begin{itemize}
    \item Write a call to a function that returns the midpoint
      of a line segment defined by $\vec{a}$ and $\vec{b}$.
    \item Write a call to a function that returns a point
      that is on the line segment that connects $\vec{a}$ to
      $\vec{b}$ and is one third of the distance from 
      $\vec{a}$ to $\vec{b}$.
    \item Write a call to a function that returns a point
      that is on the line segment that connects $\vec{a}$ to
      $\vec{b}$ and is two thirds of the distance from 
      $\vec{a}$ to $\vec{b}$.
    \end{itemize}

  \end{frame}

\begin{frame}
\frametitle{Multiplication of matrices}
\framesubtitle{$2 \times 2$ matrices}

  \begin{align*}
    \left[ \begin{array}{rr}
      a & b \\
      c & d
      \end{array} \right]
    \left[ \begin{array}{rr}
      e & f \\
      g & h
      \end{array} \right]
    & = 
    \left[ \begin{array}{rr}
      a*e + b*g & a*f + b*h \\
      c*e + d*g & c*f + d*h
      \end{array} \right]
    \end{align*}

  \end{frame}

\begin{frame}
\frametitle{Multiplication of matrices}
\framesubtitle{$2 \times 2$ matrices (continued)}

  \begin{align*}
    \left[ \begin{array}{rr}
      2 & 8 \\
      4 & 6
      \end{array} \right]
    \left[ \begin{array}{rr}
      7 & 3 \\
      1 & 5
      \end{array} \right]
    & = 
    \left[ \begin{array}{rr}
      22 & 46 \\
      34 & 42
      \end{array} \right]
    \end{align*}

  \end{frame}

\begin{frame}
\frametitle{Multiplication of matrices}
\framesubtitle{$2 \times 2$ matrices}

  \begin{align*}
    \left[ \begin{array}{rr}
      a & b \\
      c & d
      \end{array} \right]
    \left[ \begin{array}{rr}
      1 & 0 \\
      0 & 1
      \end{array} \right]
    & = 
    \left[ \begin{array}{rr}
      a & b \\
      c & d
      \end{array} \right] \\
    & = 
    \left[ \begin{array}{rr}
      1 & 0 \\
      0 & 1
      \end{array} \right]
    \left[ \begin{array}{rr}
      a & b \\
      c & d
      \end{array} \right]
    \end{align*}

  \end{frame}

\begin{frame}
\frametitle{Multiplication of matrices}
\framesubtitle{$2 \times 2$ matrices (continued)}

  What do we have here?

  \begin{align*}
    \left[ \begin{array}{rr}
      2 & 8 \\
      4 & 6
      \end{array} \right]
    \left[ \begin{array}{rr}
      \frac{6}{2 \cdot 6 - 4 \cdot 8} & -\frac{8}{2 \cdot 6 - 4 \cdot 8} \\
      -\frac{4}{2 \cdot 6 - 4 \cdot 8} & \frac{2}{2 \cdot 6 - 4 \cdot 8}
      \end{array} \right]
    & = 
    \left[ \begin{array}{rr}
      1 & 0 \\
      0 & 1
      \end{array} \right]
    \end{align*}

  \end{frame}

\begin{frame}
\frametitle{Multiplication of matrices}
\framesubtitle{$2 \times 2$ matrices}

  In general, for two matrices $\mathbf{A}$ and $\mathbf{B}$
  is it the case that $\mathbf{AB} = \mathbf{BA}$?

  \begin{align*}
    \left[ \begin{array}{rr}
      a & b \\
      c & d
      \end{array} \right]
    \left[ \begin{array}{rr}
      e & f \\
      g & h
      \end{array} \right]
    & = 
    \left[ \begin{array}{rr}
      a*e + b*g & a*f + b*h \\
      c*e + d*g & c*f + d*h
      \end{array} \right] \\
    \left[ \begin{array}{rr}
      e & f \\
      g & h
      \end{array} \right]
    \left[ \begin{array}{rr}
      a & b \\
      c & d
      \end{array} \right]
    & = 
    \left[ \begin{array}{rr}
      a*e + c*f & b*e + d*f \\
      a*g + c*h & a*g + c*f
      \end{array} \right]
    \end{align*}

  \end{frame}

\begin{frame}
\frametitle{Multiplication of matrices}
\framesubtitle{More general case}

  \begin{itemize}
    \item $\mathbf{A}$, $\mathbf{B}$, and $\mathbf{C}$
      are $N \times N$ matrices
    \item $\mathbf{A}_{ik}$ is the element in the $i^{th}$ row
      and $k^{th}$ column of $\mathbf{A}$
    \item Do you see the dot product?
    \item Which vectors are being multiplied?
    \end{itemize}

  \begin{align*}
    \mathbf{C}_{ij} & = \sum_{k = 0}^N \mathbf{A}_{ik} \mathbf{B}_{kj}
    \end{align*}

  \end{frame}

\begin{frame}[fragile]
\frametitle{Multiplication of matrices}
\framesubtitle{More general case (continued)}

  \begin{lstlisting}{}
  for( int j = 0; j < N; j++ ) {
    for( int i = 0; i < N; i++ ) {
      C[i][j] = 0;
      for( int k = 0; k < N; k++ ) {
        C[i][j] += A[i][k] * B[k][j];
      } // for
    } // for
  } // for
    \end{lstlisting}

\end{frame}

\begin{frame}
\frametitle{Affine transformations}
\framesubtitle{$2 \times 2$ operators}

  \begin{align*}
    \mathbf{R} & =
    \left[ \begin{array}{rr}
      \cos \psi & -\sin \psi \\
      \sin \psi & \cos \psi
      \end{array} \right] \\
    \mathbf{R}^T & =
    \left[ \begin{array}{rr}
      \cos \psi & \sin \psi \\
      -\sin \psi & \cos \psi
      \end{array} \right] \\
    & =
    \left[ \begin{array}{rr}
      \cos (-\psi) & -\sin (-\psi) \\
      \sin (-\psi) & \cos (-\psi)
      \end{array} \right] \\
    \mathbf{R}\mathbf{R}^T & =
    \left[ \begin{array}{rr}
      1 & 0 \\
      0 & 1 
      \end{array} \right] \\
    & = \mathbf{R}^T \mathbf{R}
    \end{align*}

  \end{frame}

\begin{frame}
\frametitle{Affine transformations}
\framesubtitle{$2 \times 2$ operators}

  What is the geometric interpretation?

  \begin{align*}
    \left[ \begin{array}{rr}
      \cos \frac{\pi}{2} & -\sin \frac{\pi}{2} \\
      \sin \frac{\pi}{2} &  \cos \frac{\pi}{2}
      \end{array} \right]
    \left[ \begin{array}{r}
      1 \\
      0
      \end{array} \right]
    & = 
    \left[ \begin{array}{r}
      0 \\
      1
      \end{array} \right] \\
    \left[ \begin{array}{rr}
      \cos \frac{\pi}{2} & -\sin \frac{\pi}{2} \\
      \sin \frac{\pi}{2} &  \cos \frac{\pi}{2}
      \end{array} \right]
    \left[ \begin{array}{r}
      1 \\
      1
      \end{array} \right]
    & = 
    \left[ \begin{array}{r}
      -1 \\
      1
      \end{array} \right]
    \end{align*}

  \end{frame}

\begin{frame}
\frametitle{Affine transformations}
\framesubtitle{$2 \times 2$ operators}

  What is the geometric interpretation?

  \begin{align*}
    \left[ \begin{array}{rr}
      \cos \frac{\pi}{4} & -\sin \frac{\pi}{4} \\
      \sin \frac{\pi}{4} &  \cos \frac{\pi}{4}
      \end{array} \right]
    \left[ \begin{array}{r}
      1 \\
      0
      \end{array} \right]
    & = 
    \left[ \begin{array}{r}
      \sqrt{2}/2 \\
      \sqrt{2}/2
      \end{array} \right] \\
    \left[ \begin{array}{rr}
      \cos \frac{\pi}{4} & -\sin \frac{\pi}{4} \\
      \sin \frac{\pi}{4} &  \cos \frac{\pi}{4}
      \end{array} \right]
    \left[ \begin{array}{r}
      1 \\
      1
      \end{array} \right]
    & = 
    \left[ \begin{array}{r}
      0 \\
      \sqrt{2}  
      \end{array} \right]
    \end{align*}

  \end{frame}

\begin{frame}
\frametitle{Affine Transformations}
\framesubtitle{$3 \times 3$ matrices but still in a 2D world!}

  How does this operator transform the vector? Scales it.

  \begin{align*}
    \left[ \begin{array}{rrr}
      \sigma_x & 0 & 0 \\
      0 & \sigma_y & 0 \\
      0 & 0 & 1
      \end{array} \right]
    \left[ \begin{array}{r}
      v_x \\
      v_y \\
      1
      \end{array} \right]
    & =
    \left[ \begin{array}{r}
      v_x * \sigma_x  \\
      v_y * \sigma_y  \\
      1
      \end{array} \right]
    \end{align*}

  \end{frame}

\begin{frame}
\frametitle{Affine Transformations}
\framesubtitle{$3 \times 3$ matrices but still in a 2D world!}

  How does this operator transform the vector? Translates it.

  \begin{align*}
    \left[ \begin{array}{rrr}
      1 & 0 & \delta_x \\
      0 & 1 & \delta_y \\
      0 & 0 & 1
      \end{array} \right]
    \left[ \begin{array}{r}
      v_x \\
      v_y \\
      1
      \end{array} \right]
    & =
    \left[ \begin{array}{r}
      v_x + \delta_x  \\
      v_y + \delta_y  \\
      1
      \end{array} \right]
    \end{align*}

  \end{frame}

\end{document}
