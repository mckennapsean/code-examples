% by Sean McKenna, Suzannah Wood, and John Klingner

\documentclass{beamer}
\usepackage[latin1]{inputenc}
\usepackage{chemfig}
\usetheme{Warsaw}
\title[Quantum Processing with NMR]{Quantum Information Processing\\using Nuclear Magnetic Resonance}
\author{John Klingner, Sean McKenna, Suzannah Wood}
\institute{Cornell College}
%	\date{Decemeber  19, 2011}
\bibliographystyle{plain}
\usepackage{subfig}
\usepackage{wrapfig}

\begin{document}

\begin{frame}
\titlepage
\end{frame}


\begin{frame}{Quantum Computing}
	\begin{itemize}
	\item Computers - store and process information 
		\begin{itemize}
			\item quantum bit (qubit)
		\end{itemize}
	\vspace{1em}
	\item potential for faster algorithms
	\vspace{1em}
	\item limited qubits
	\item nuclear magnetic resonance (NMR) spectroscopy
	\vspace{1em}
%	\item Methods of quantum computing
%		\begin{itemize}
%		\item ion-trap experiments
%		\item quantum dots
%		\item nuclear magnetic resonance (NMR) spectroscopy
%		\end{itemize}
	\end{itemize}
\end{frame}

\begin{frame}{NMR and Quantum Computing}
	\begin{itemize}
	\item Intrinsic angular momentum\cite{QMbook}
		\begin{itemize}
		\item fermions: spin $\frac{1}{2}$
			\begin{itemize}
			\item e.g. protons, neutrons, electrons
			\end{itemize}
		\item magnetic moment: aligns with a magnetic field
		\end{itemize}
	\end{itemize}
	\begin{figure}	
	\includegraphics[width=0.7\textwidth]{data/magneticfield.png}
	\end{figure}
\end{frame}

\begin{frame}{NMR and Quantum Computing}
	\begin{itemize}
	\item Strong super conducting magnet \cite{ABC's}
	\vspace{1em}
	\item Sample NMR tube \cite{NMRprep}
	\vspace{1em}
	\item Electromagnetic pulse
	\begin{itemize}
		\item at resonance, ``tip'' the nucleus
		\item affected by pulse shapes and widths
		\item combine to form logic gates 
	\end{itemize}
	\end{itemize}
\end{frame}

\begin{frame}{Free Induction Decay and Fourier Transform}
	\begin{figure}
	\begin{minipage}[c]{.2\textwidth}
		\caption*{Free Induction Decay}
	\end{minipage}%
	\begin{minipage}[c]{.8\textwidth}
		\includegraphics[height=0.4\textheight]{data/FID.png}
	\end{minipage}
	\end{figure}
	\begin{figure}
	\begin{minipage}[c]{.2\textwidth}
		\caption*{Fourier Transform}
	\end{minipage}%
	\begin{minipage}[c]{.8\textwidth}
		\includegraphics[height=0.4\textheight]{data/FFT.png}
	\end{minipage}
	\end{figure}
\end{frame}

\begin{frame}{Quantum System - Sample}	
	\begin{itemize}	
	\item Chloroform: $^{13}$C isotope ($^{13}$CHCl$_{3}$)
		\begin{itemize}
		\item $^{13}$C: spin $\frac{1}{2}$
		\item $^{1}$H: spin $\frac{1}{2}$
		\vspace{1em}
		\chemfig{\llap{${}^{13}$}C(-[5]Cl)(-[2]H)(<[:-70]Cl)(<:[:-20]Cl)}
		\vspace{1em}
		\item spins align within the magnetic field 
			\begin{itemize}
			\item four possible states
			\end{itemize}
		\end{itemize}
		\item two-qubit system
	\end{itemize}
\end{frame}

\begin{frame}{``Pure'' States}
	\begin{itemize}
	\item semi-effective ``pure'' state: 0 K
	\vspace{1em}
	\item temporal averaging: add results to cancel out spins
	\begin{itemize}
		\item two qubits results in three experiments
	\end{itemize}
	\vspace{1em}
	\item four input spin states
	\end{itemize}
\end{frame}

\begin{frame}{Thermal vs. ``Pure'' State}
	\begin{figure}[htb]
	\begin{minipage}{0.8\textwidth}
		\begin{centering}
	\begin{figure}
		\subfloat{\includegraphics[height=0.4\textheight]{data/00prepT.png}}
		\\
		\subfloat{\includegraphics[height=0.4\textheight]{data/00prep.png}}
	\end{figure}
		\end{centering}
	\end{minipage}%
	\begin{minipage}[t]{0.2\textwidth}
	\begin{centering}
	\begin{tabular}{c | c}
		Qubits & Spin \\
		\hline
		00 & $\uparrow \uparrow$ \\
		01 & $\uparrow \downarrow$ \\
		10 & $\downarrow \uparrow$ \\
		11 & $\downarrow \downarrow$ \\
	\end{tabular}
	\end{centering}
	\end{minipage}%
	\end{figure}

\end{frame}

\begin{frame}{Pulse Sequences}
	\begin{itemize}
	\item translation of charts from literature
	\begin{itemize}
		\item hardware \& software limitations
	\end{itemize}
	\item sequences may vary
	\item pulse sequences for temporal averaging\cite{bigdaddy}
	\end{itemize}
	\begin{figure}
		\includegraphics[width=0.7\textwidth]{data/bigdaddy.png}
	\end{figure}
\end{frame}

\begin{frame}{Pulse Axes}
	\begin{figure}
	\begin{minipage}[c]{.2\textwidth}
		\caption*{x90 Pulse}
	\end{minipage}%
	\begin{minipage}[c]{.8\textwidth}
		\includegraphics[height=0.4\textheight]{data/x-pulse.png}
	\end{minipage}
	\end{figure}
	\begin{figure}
	\begin{minipage}[c]{.2\textwidth}
		\caption*{y90 Pulse}
	\end{minipage}%
	\begin{minipage}[c]{.8\textwidth}
		\includegraphics[height=0.4\textheight]{data/y-pulse.png}
	\end{minipage}
	\end{figure}
\end{frame}

\begin{frame}{Preparation of Four States}
	\begin{figure}
	\begin{minipage}[c]{.2\textwidth}
		\caption*{00 | $\uparrow \uparrow$}
	\end{minipage}%
	\begin{minipage}[c]{.8\textwidth}
		\includegraphics[height=0.4\textheight]{data/00prep.png}
	\end{minipage}
	\end{figure}
	\begin{figure}
	\begin{minipage}[c]{.2\textwidth}
		\caption*{01 | $\uparrow \downarrow$}
	\end{minipage}%
	\begin{minipage}[c]{.8\textwidth}
		\includegraphics[height=0.4\textheight]{data/01prep.png}
	\end{minipage}
	\end{figure}
\end{frame}

\begin{frame}{Preparation of Four States}
	\begin{figure}
	\begin{minipage}[c]{.2\textwidth}
		\caption*{00 | $\uparrow \uparrow$}
	\end{minipage}%
	\begin{minipage}[c]{.8\textwidth}
		\includegraphics[height=0.4\textheight]{data/00prep.png}
	\end{minipage}
	\end{figure}
	\begin{figure}
	\begin{minipage}[c]{.2\textwidth}
		\caption*{10 | $\downarrow \uparrow$}
	\end{minipage}%
	\begin{minipage}[c]{.8\textwidth}
		\includegraphics[height=0.4\textheight]{data/10prep.png}
	\end{minipage}
	\end{figure}
\end{frame}

\begin{frame}{Preparation of Four States}
	\begin{figure}
	\begin{minipage}[c]{.2\textwidth}
		\caption*{00 | $\uparrow \uparrow$}
	\end{minipage}%
	\begin{minipage}[c]{.8\textwidth}
		\includegraphics[height=0.4\textheight]{data/00prep.png}
	\end{minipage}
	\end{figure}
	\begin{figure}
	\begin{minipage}[c]{.2\textwidth}
		\caption*{11 | $\downarrow \downarrow$}
	\end{minipage}%
	\begin{minipage}[c]{.8\textwidth}
		\includegraphics[height=0.4\textheight]{data/11prep.png}
	\end{minipage}
	\end{figure}
\end{frame}

\begin{frame}{Preparation of Four States}
	\vspace{-2em}
	\begin{figure}
		\subfloat{\includegraphics[height=0.3\textheight]{data/00prep.png}}
		\subfloat{\includegraphics[height=0.3\textheight]{data/01prep.png}}
		\\
		\subfloat{\includegraphics[height=0.3\textheight]{data/10prep.png}}
		\subfloat{\includegraphics[height=0.3\textheight]{data/11prep.png}}
	\end{figure}
\end{frame}

\begin{frame}{C-Not Gate}
	\begin{itemize}
	\item with initial states, implemented controlled-not (C-Not) gate \cite{cnot}
	\begin{itemize}
		\item flips B when A = 1
	\end{itemize}
	\end{itemize}
	\begin{figure}[htb]
	\begin{minipage}{0.3\textwidth}
		\begin{centering}
	\begin{figure}
	\begin{tabular}{c c | c c}
	\multicolumn{2}{c|}{In} & \multicolumn{2}{|c}{Out} \\
	A & B & A & B \\
	\hline
	0 & 0 & 0 & 0 \\
	0 & 1 & 0 & 1 \\
	1 & 0 & 1 & 1 \\
	1 & 1 & 1 & 0 \\
	\end{tabular}
	\caption*{Truth Table for C-Not Gate}
	\end{figure}
		\end{centering}
	\end{minipage}%
	\begin{minipage}[t]{0.7\textwidth}
	\begin{centering}
		\begin{figure}
	\includegraphics[width=0.7\textwidth]{data/controllednot.png}
	\end{figure}
	\end{centering}
	\end{minipage}%
	\end{figure}
\end{frame}

\begin{frame}{C-Not Gate: Results}
	\vspace{-2em}
	\begin{figure}
		\subfloat{\includegraphics[height=0.3\textheight]{data/00cNot.png}}
		\subfloat{\includegraphics[height=0.3\textheight]{data/01cNot.png}}
		\\
		\subfloat{\includegraphics[height=0.3\textheight]{data/10cNot.png}}
		\subfloat{\includegraphics[height=0.3\textheight]{data/11cNot.png}}
	\end{figure}
\end{frame}

\begin{frame}{Conclusion}
	\begin{itemize}
	\item demonstrated quantum NMR
	\item prepared four initial states
	\item nearly completed a logic gate
	\vspace{1em}
	\item future work
		\begin{itemize}
		\item understand, create, and simplify pulse sequences
		\item run quantum algorithms
		\item expand beyond two qubits
		\end{itemize}
	\end{itemize}
\end{frame}

\begin{frame}{References}
\bibliography{bibfileQM}
\end{frame}

\begin{frame}{Full Size FID}
	\includegraphics[width=1\textwidth]{data/FID.png}
\end{frame}

\begin{frame}{Full Size FFT}
	\includegraphics[width=1\textwidth]{data/FFT.png}
\end{frame}

\begin{frame}{Full Size x90}
	\includegraphics[width=1\textwidth]{data/x-pulse.png}
\end{frame}

\begin{frame}{Full Size y90}
	\includegraphics[width=1\textwidth]{data/y-pulse.png}
\end{frame}

\begin{frame}{Full Size 00prep}
	\includegraphics[width=1\textwidth]{data/00prep.png}
\end{frame}

\begin{frame}{Full Size 01prep}
	\includegraphics[width=1\textwidth]{data/01prep.png}
\end{frame}

\begin{frame}{Full Size 10prep}
	\includegraphics[width=1\textwidth]{data/10prep.png}
\end{frame}

\begin{frame}{Full Size 11prep}
	\includegraphics[width=1\textwidth]{data/11prep.png}
\end{frame}

\begin{frame}{Full Size 00cNot}
	\includegraphics[width=1\textwidth]{data/00cNot.png}
\end{frame}

\begin{frame}{Full Size 01cNot}
	\includegraphics[width=1\textwidth]{data/01cNot.png}
\end{frame}

\begin{frame}{Full Size 10cNot}
	\includegraphics[width=1\textwidth]{data/10cNot.png}
\end{frame}

\begin{frame}{Full Size 11cNot}
	\includegraphics[width=1\textwidth]{data/11cNot.png}
\end{frame}

\end{document}
