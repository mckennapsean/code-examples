% by Sean McKenna

\documentclass{beamer}
\usetheme{Madrid} % My favorite!
%\usetheme{Boadilla} % Pretty neat, soft color.
%\usetheme{default}
%\usetheme{Warsaw}
%\usetheme{Bergen} % This template has nagivation on the left
%\usetheme{Frankfurt} % Similar to the default 
%with an extra region at the top.
%\usecolortheme{seahorse} % Simple and clean template
%\usetheme{Darmstadt} % not so good
% Uncomment the following line if you want %
% page numbers and using Warsaw theme%
% \setbeamertemplate{footline}[page number]
%\setbeamercovered{transparent}
\setbeamercovered{invisible}
% To remove the navigation symbols from 
% the bottom of slides%
\setbeamertemplate{navigation symbols}{} 
%
\usepackage{graphicx}
%\usepackage{bm} % For typesetting bold math (not \mathbold)
%\logo{\includegraphics[height=0.6cm]{yourlogo.eps}}
%
\title{Galaxy Mergers and Star Formation using TreeSPH}
\author{Sean McKenna}
\institute[Cornell College]
{ Cornell College \\
\medskip
{\emph{smckenna12@cornellcollege.edu}}
}
\date{\today}
% \today will show current date. 
% Alternatively, you can specify a date.
%
\begin{document}
%
\begin{frame}
\titlepage
\end{frame}

\begin{frame}
  \frametitle{Introduction}
  Let's start with some animations...
  \\ http://www.cfa.harvard.edu/seuforum/animations/animations
  \\ http://www.galaxydynamics.org/tflops
\end{frame}

\begin{frame}
  \frametitle{Introduction}
  \begin{itemize}
    \item Milky Way collides with Andromeda!
    \item three billion year countdown
    \item two billion years before the Sun as red giant
    \item collision = interaction
    \item night sky: dark or light
    \item numerical simulations
  \end{itemize}
\end{frame}

\begin{frame}
  \frametitle{Introduction}
  \begin{block}{Thesis}
    Computational astrophysicists have developed efficient algorithms for modeling galactic interactions which enable researchers to find an increase in the rate of star formation during these encounters.
  \end{block}
\end{frame}

\begin{frame}
  \frametitle{Concepts - Numerical Simulations}
  \begin{itemize}
    \item formulation of models
    \item semi-analytic technique
    \item balance:
    \begin{itemize}
      \item accuracy
      \item computational time
    \end{itemize}
    \item algorithms:
    \begin{itemize}
      \item procedural, step-by-step
      \item mathematical tricks
    \end{itemize}
  \end{itemize}
\end{frame}

\begin{frame}
  \frametitle{Concepts - Galaxies}
  \begin{figure}
    \includegraphics[scale=0.235]{galaxies.jpg}
  \end{figure}
\end{frame}

\begin{frame}
  \frametitle{Concepts - Galaxies}
  \begin{itemize}
    \item collection of stars, dust, and gas
    \item held together by gravity
    \item types
    \begin{itemize}
      \item cluster
      \item dwarf
      \item elliptical
      \item irregular
      \item spiral
    \end{itemize}
    \item gas: up to 10\%
  \end{itemize}
\end{frame}

\begin{frame}
  \frametitle{Concepts - Dark Matter}
  \begin{itemize}
    \item discovered by Zwicky
    \item anomalous observations of gravitational shifting of objects
    \item gravitational interactions
    \item no electromagnetic interactions, etc.
    \item play a role in models of galaxies
    \item average density calculated
  \end{itemize}
\end{frame}

\begin{frame}
  \frametitle{Concepts - Star Formation}
  \begin{figure}
    \includegraphics[scale=0.46]{arp220.jpg}
  \end{figure}
\end{frame}

\begin{frame}
  \frametitle{Concepts - Star Formation}
  \begin{itemize}
    \item Star formation rate (SFR):
    \begin{itemize}
      \item how quickly stars form in a galaxy
    \end{itemize}
    \item Star formation efficiency (SFE):
    \begin{itemize}
      \item governed by: $SFE(t) = \frac{M_{gas \to \star}(t)}{M_{gas}(t)}$
    \end{itemize}
    \item starburst galaxies (starbursts): high SFR
    \begin{itemize}
      \item Arp220
    \end{itemize}
    \item hint to the past and future
  \end{itemize}
\end{frame}

\begin{frame}
  \frametitle{Concepts - Galaxy Interactions \& Mergers}
  \begin{figure}
    \includegraphics[scale=0.12]{ngc4676.jpg}
  \end{figure}
\end{frame}

\begin{frame}
  \frametitle{Concepts - Galaxy Interactions \& Mergers}
  \begin{itemize}
    \item gravitational and tidal interactions
    \item stars form from dense areas of gas
    \item mergers:
    \begin{itemize}
      \item aka, collision
      \item spiral + spiral = elliptical
      \item MWG + AG
    \end{itemize}
  \end{itemize}
\end{frame}

\begin{frame}
  \frametitle{Model - Purpose}
  \begin{itemize}
    \item reduce complex problem (semi-analytic)
    \item allow problem to be solved numerically
    \item simplify, reduces accuracy
    \item must make assumptions
  \end{itemize}
\end{frame}

\begin{frame}
  \frametitle{Model - Assumptions \& Variables}
  \begin{itemize}
    \item spherical galaxies
    \item dark matter in the galaxies
    \item known density gradient
    \item mass ratios
    \item isothermal gas
    \item $N = 120,000$
    \item $T = 10^4 $K
    \item variables:
    \begin{itemize}
      \item separation distance
      \item spin orientation
      \item galaxy type
      \item mass ratio
      \item ages
      \item angle of interaction
    \end{itemize}
  \end{itemize}
\end{frame}

\begin{frame}
  \frametitle{Numerical Simulations}
  \begin{itemize}
    \item algorithm: TreeSPH
    \item TreeSPH: tree hierarchical method and smoothed particle hydrodynamics
    \item method is Lagrangian (both pieces)
    \item tree hierarchical method
    \begin{itemize}
      \item cluster stars together
      \item get cluster gravitational interactions
      \item stores clusters in a tree data structure
    \end{itemize}
    \item smoothed particle hydrodynamics (SPH)
    \begin{itemize}
      \item important role of gas in galaxies
      \item coupled with the gravitational interactions
    \end{itemize}
    \item together, decent running time and semi-accurate results
  \end{itemize}
\end{frame}

\begin{frame}
  \frametitle{Results}
  \begin{figure}
    \includegraphics[scale=0.37]{merger.jpg}
  \end{figure}
\end{frame}

\begin{frame}
  \frametitle{Results}
  \begin{itemize}
    \item interactions and mergers may not affect SFR
    \item mergers, increased SFR
    \item retrograde mergers, increased SFR
    \item amount of starting gas, not parameter for SFE
    \item 15\% of interactions increased SFR by a multiple of five or more
    \item merging time, lasts about $10^8$ years
  \end{itemize}
\end{frame}

\begin{frame}
  \frametitle{GalMer Project}
  \begin{itemize}
    \item online website for storage and retrieval of galactic merger simulations
    \item over 1,000 simulations online currently
    \item virtual observatory hosted in France
    \item tools for observers:
    \begin{itemize}
      \item FITS tables
      \item IFU spectra
      \item color-magnitude plots
      \item FITS images
      \item velocity dispersion graph
      \item gas density graph
      \item stellar density graph
      \item stellar metallicity graph
      \item and more...
    \end{itemize}
  \end{itemize}
\end{frame}

\begin{frame}
  \frametitle{GalMer Project}
  \begin{figure}
    \includegraphics[scale=0.55]{galmer1.jpg}
  \end{figure}
\end{frame}

\begin{frame}
  \frametitle{GalMer Project}
  \begin{figure}
    \includegraphics[scale=0.43]{galmer2.jpg}
  \end{figure}
\end{frame}

\begin{frame}
  \frametitle{GalMer Project}
  \begin{figure}
    \includegraphics[scale=0.39]{aladin.jpg}
  \end{figure}
\end{frame}

\begin{frame}
  \frametitle{GalMer Project}
  \begin{figure}
    \includegraphics[scale=0.42]{vospec.jpg}
  \end{figure}
\end{frame}

\begin{frame}
  \frametitle{GalMer Project}
  \begin{figure}
    \includegraphics[scale=0.8]{votools.jpg}
  \end{figure}
\end{frame}

\begin{frame}
  \frametitle{Conclusion}
  \begin{itemize}
    \item astrophysicists use numerical simulations
    \item computations can be very complex
    \item supercomputers: tools required to process these kinds of simulations
    \item models: balance between accuracy and simplicity
    \item merging galaxies see increases of SFR
  \end{itemize}
\end{frame}

\begin{frame}
  \frametitle{References}
    http://www.cfa.harvard.edu/seuforum/animations/
    \\ http://www.galaxydynamics.org/tflops.html
    \\ http://www.ipac.caltech.edu/2mass/gallery/galmorph/nir\_morph.htm
    \\ http://apod.nasa.gov/apod/ap970617.html
    \\ http://apod.nasa.gov/apod/ap080224.html
    \\ http://www.journal-therapie.org/10.1051/0004-6361:20066959
    \\ http://galmer.obspm.fr/
\end{frame}

\end{document}
