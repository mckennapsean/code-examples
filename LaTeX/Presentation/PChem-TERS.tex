% by Sean McKenna

\documentclass{beamer}
\usetheme{Madrid}
\setbeamercovered{invisible}
\setbeamertemplate{navigation symbols}{}
\usepackage{graphicx}
\usepackage{hyperref}
\usepackage[sort,compress,comma,super]{natbib}
\bibliographystyle{apalike}

%%% CUSTOM COMMANDS
% define commands for super and sub scripts in text
\newcommand{\super}[1]{\ensuremath{^{\textrm{#1}}}}
\newcommand{\sub}[1]{\ensuremath{_{\textrm{#1}}}}
%for natbib & beamer to kill error
\def\newblock{\hskip .11em plus .33em minus .07em}

\title{Mapping of B\sub{x}N\sub{y}C\sub{z}: Utilizing TERS to Analyze \textit{h}-BNC}
\author{Sean McKenna}
\institute[Cornell College]{ Cornell College \\ \medskip {\emph{smckenna12@cornellcollege.edu}} }
\date{\today}


\begin{document}


\begin{frame}
  \titlepage
\end{frame}

\begin{frame}
  \frametitle{Introduction}
  \begin{itemize}
    \item graphene and \textit{h}-BN each have intriguing properties\cite{main,bn}
    \item while quite similar, there are key differences between the two:
    \begin{itemize}
      \item Van der Waals forces
      \item ring stacking
      \item charge distribution
    \end{itemize}
    \item graphene antidot lattices (GALs) show semiconductor behavior\cite{qubit}
    \item GALs are weak, \textit{h}-BNC or B\sub{x}N\sub{y}C\sub{z} improves this\cite{main}
    \item tunable band gaps in a semiconductor can be used in next-gen nanotechnology and nanodevices
    \item other possible applications include: nano-optics, quantum computing
  \end{itemize}
\end{frame}

\begin{frame}
  \frametitle{Background - Very Basic Semiconductor Physics}
  \begin{figure}
    \includegraphics[scale=1.5]{main.png}
    \caption{Electronic band structures of several structures, scaled to Fermi energy\cite{main}}
  \end{figure}
  \begin{itemize}
    \item band gap: difference in energy of the conduction and valence bands
  \end{itemize}
\end{frame}

\begin{frame}
  \frametitle{Background - Illustration}
  \begin{figure}
    \includegraphics[scale=0.6]{h-bnc.png}
    \caption{Different lattice formations: a. graphene antidot lattice (GAL),\cite{qubit} b. \textit{h}-BN,\cite{bn} c. experimental hypothesized \textit{h}-BNC,\cite{synth} and d. theoretical \textit{h}-BNC.\cite{main}}
  \end{figure}
\end{frame}

\begin{frame}
  \frametitle{Background - Synthesis\cite{synth}}
  \begin{itemize}
    \item mixed methane (CH\sub{4}) and ammonia borane (BH\sub{3}NH\sub{3})
    \item process of thermal catalytic chemical vapor deposition (CVD)
    \item resulted in segregated domains of graphene and \textit{h}-BN
    \item supported their hypothesis with: AFM, HR-TEM, EELS, FFT, XPS, Raman, UV-vis, and electrodes
    \item tunable band gaps depending on the amount of carbon present
    \item actual structure was not identified, only distribution
  \end{itemize}
\end{frame}

\begin{frame}
  \frametitle{Background - Theoretical Study\cite{main}}
  \begin{itemize}
    \item DFT calculations of several segregated domains of \textit{h}-BN
    \item tested two rhombus shapes, two triangular formations, and a hexagon
    \item larger domains are more thermodynamically favorable
    \item conducted tests again with a more precise basis set, found agreement
    \item also found tunable band gaps dependent on nanodomain size
    \item assumed simple nanodomains based on prior work
  \end{itemize}
\end{frame}

\begin{frame}
  \frametitle{Background - Tip-Enhanced Raman Spectroscopy (TERS)\cite{ters}}
  \begin{itemize}
    \item similar to combining AFM and SERS
    \item generates high-resolution spectra
    \item can analyze chemical species on the nanoscale
    \item certain fine AFM tips can provide lateral resolution down to 0.2 nm
  \end{itemize}
\end{frame}

\begin{frame}
  \frametitle{Background - Tip-Enhanced Raman Spectroscopy (TERS)}
  \begin{figure}
    \includegraphics[scale=0.3]{ters.png}
    \caption{Illustration of the TERS method\cite{ters}}
  \end{figure}
\end{frame}

\begin{frame}
  \frametitle{Proposal}
  \begin{block}{Fundamental Research Question}
    What is the actual structure of the \textit{h}-BNC lattice?
  \end{block}
  \begin{block}{Significance}
    Research, understanding, modeling, manufacturing
  \end{block}
\end{frame}

\begin{frame}
  \frametitle{Proposal - Mapping}
  \begin{itemize}
    \item use TERS to properly scan and map \textit{h}-BNC
    \item analyze the spectra to find what atoms are most dominant in a region
    \item like putting a puzzle together:
    \begin{itemize}
      \item predominantly carbon: graphene domain
      \item predominantly boron \& nitrogen: \textit{h}-BN domain
      \item some mixture: edge or meeting of two (or more) domains
    \end{itemize}
    \item find the size, shape, and structure of the different nanodomains
  \end{itemize}
\end{frame}

\begin{frame}
  \frametitle{Proposal - Modeling}
  \begin{itemize}
    \item use DFT to verify experimental results
    \item conduct a theoretical study based on the actual sizes and shapes of nanodomains
    \item compare all results to this study's experiment and other studies too
    \item seek to add kinetics into the model to better predict shape and size
  \end{itemize}
\end{frame}

\begin{frame}
  \frametitle{References}
  \bibliography{sources}
\end{frame}


\end{document}
