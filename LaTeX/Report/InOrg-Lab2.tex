% by Sean McKenna

% This is a LaTeX version of a laboratory report
\documentclass[11pt]{article}

% necessary imports
\usepackage{enumerate}
\usepackage{amsmath}
\usepackage{amssymb}
\usepackage{makeidx}
\usepackage{graphicx}
\usepackage{hyperref}
\usepackage[utf8]{inputenc}
\usepackage{geometry}
\usepackage{booktabs}
\usepackage{array}
\usepackage{verbatim}
\usepackage{subfig}
\usepackage[super]{natbib}
\usepackage{mciteplus}

% page adjustments
\setlength{\parindent}{0pt}
\geometry{a4paper}

% expand margins?
%\addtolength{\oddsidemargin}{-.5in}
%\addtolength{\evensidemargin}{-.5in}
%\addtolength{\textwidth}{1in}
%\addtolength{\topmargin}{-.5in}
%\addtolength{\textheight}{1in}

% double-spaced?
%\linespread{2}

%define commands for super and sub scripts in text
\newcommand{\super}[1]{\ensuremath{^{\textrm{#1}}}}
\newcommand{\sub}[1]{\ensuremath{_{\textrm{#1}}}}

% title
\title{Determination of $\Delta$\sub{o} in Cr(III) Complexes}
\author{Sean McKenna}
\date{\today}

% begin the document
\begin{document}
\maketitle

% header info
\begin{center}
\begin{tabular}{lr}
Dates Performed: March 5, 7, and 13, 2012 & Lab Partner: Jose Martinez\\
Instructor: Professor Cindy Strong
\end{tabular}
\end{center}


% describe the experiment's method
\section{Method}
Following a predetermined method,\cite{lab} the spectra of several chemicals along a spectrochemical series were taken such that the wavelengths, and thus associated energy differences, could be compared. \\

While transferring the CrCl\sub{3}$\cdot$6H\sub{2}O from the weigh boat into the 5 mL round-bottom flask, the flakes stuck to the weigh boat due to electrostatic forces. As such, some of the mass was lost in transferring the powder into the solution. \\

Misreading the Pasteur pipet, only 10 $\mu$L of both ethylenediamine and methanol were added to the mixture before the solution was refluxed for approximately seven minutes. After that time, the refluxing was stopped to add an additional 990 $\mu$L of both solutions to total 1 mL of each solution in the mixture. Then, refluxing restarted with the full mixture for a full hour. Adding the volumes in parts increases the uncertainty in each of the volumes. \\

For rinsing the filter paper during the suction filtration, a 10\% by weight solution of ethylenediamine and methanol was prepared by using their respective densities and a set volume. After doing so, a blue-gray powder was left mixed in with the zinc. It was difficult to determine the difference between the two, but the product had a ``caked mud'' consistency that differentiated itself from the zinc with careful consideration. Complete separation was not achieved, and a percent yield was thus not feasible because of the mixed product. \\

To obtain a clear UV-vis spectrum, the product was mixed in water and pushed through a syringe with a 0.20$\mu$m filter to get a clear (not murky) solution. Using the Agilant 8453 UV-visible spectrophotometer, a spectrum was obtained. \\

Working with Elizabeth MacGregor and Laura Kelton, the other three spectra were observed on the Cary 50 Bio UV-visible spectrophotometer. These spectra are also attached. \\

Lastly, for the spectra of one compound taken over time, the Agilant device was also used, taking initial measurements, 1.3 hour, 2.3 hour, 3.5 hour, 6.0 hour, and 25.0 hour measurements of a spectrum over time. These are the final spectra attached.


% show the reaction/s
\section{Reactions}
\begin{center}
CrCl\sub{3}$\cdot$6H\sub{2}O (s) + 3[C\sub{2}H\sub{4}(NH\sub{2})\sub{2}] (aq) $\rightarrow$ C\sub{6}H\sub{24}Cl\sub{3}CrN\sub{6} (s) + 6H\sub{2}O (l)\\
\textit{or} \\
CrCl\sub{3}$\cdot$6H\sub{2}O + 3[en] $\rightarrow$ [Cr(en)\sub{3}]Cl\sub{3} + 6H\sub{2}O
\end{center}


% show some calculations
\section{Sample Calculations}
The wavelengths were extracted from the peaks in the spectra, and these were used to calculate the energy between the two split d-orbital energy levels ($\Delta$\sub{o}). \\

\begin{center}
  $\Delta_{o} \, = \, \nu \, = \, \frac{1 \times 10^{7} \textrm{nm/cm}}{\lambda \textrm{ (in nm)}} \, = \, \frac{1 \times 10^{7} \textrm{nm/cm}}{485 \textrm{ nm}} \, = \, 2.06 \times 10^{4} \textrm{ cm}^{-1}$
\end{center}


% show any final results from the lab, tabular form
\section{Results}
The following results were found from the spectra that were measured. The spectrum of the compound that was created in this lab is in Figure 1. Spectra of the other compounds by another lab group is in Figure 2.

\begin{center}
\begin{tabular}{|l|c|c|}
\hline
\textbf{Solution} & \textbf{$\lambda$\sub{max} (nm)} & \textbf{$\Delta$\sub{o} (cm\super{-1})} \\
\hline
$[$Cr(H\sub{2}O)\sub{4}Cl\sub{2}$]$ Cl$\cdot$2H\sub{2}O or CrCl\sub{3}$\cdot$6H\sub{2}O & 631 & $1.59 \times 10^{4}$ \\
$[$Cr(H\sub{2}O)\sub{6}$]$(NO\sub{3})\sub{3}$\cdot$3H\sub{2}O & 576  & $1.74 \times 10^{4}$ \\
$[$Cr(acac)\sub{3}$]$ & 560 & $1.79 \times 10^{4}$ \\
$[$Cr(en)\sub{3}$]$Cl\sub{3}$\cdot$3H\sub{2}O & 485 & $2.06 \times 10^{4}$ \\
\hline
\end{tabular}
\end{center}

Additionally, the results of the UV-spectra of CrCl\sub{3}$\cdot$6H\sub{2}O over time were recorded in Figure 3, as summarized in the table below.

\begin{center}
\begin{tabular}{|c|c|c|}
\hline
\textbf{Time (hours)} & \textbf{$\lambda$\sub{max} (nm)} & \textbf{$\Delta$\sub{o} (cm\super{-1})} \\
\hline
0 & 631 & $1.59 \times 10^{4}$ \\
1.3 & 596 & $1.68 \times 10^{4}$ \\
2.3 & 588 & $1.70 \times 10^{4}$ \\
3.5 & 581 & $1.72 \times 10^{4}$ \\
6.0 & 578 & $1.73 \times 10^{4}$ \\
25.0 & 578 & $1.73 \times 10^{4}$ \\
\hline
\end{tabular}
\end{center}


% discuss what was found, compare to literature, answer specified questions
\section{Discussion}
Comparing to other reported values, a table\cite{site} was found (which is likely from one of the references on the bottom of the website) displaying three out of the four electronic spectra transitions corresponding to $\Delta$\sub{o} for these complexes. There is good agreement with the values presented in the table below and the ones measured in lab. One complex was not in this resource, but it will be shown that its value is in the proper placement in the series.

\begin{center}
\begin{tabular}{|l|c|c|}
\hline
\textbf{Complex} & \textbf{measured - $\Delta$\sub{o} (cm\super{-1})} & \textbf{table\cite{site} - $\Delta$\sub{o} (cm\super{-1})} \\
\hline
$[$Cr(H\sub{2}O)\sub{4}Cl\sub{2}$]$\super{1+} & $1.59 \times 10^{4}$ & --- \\
$[$Cr(H\sub{2}O)\sub{6}$]$\super{3+} & $1.74 \times 10^{4}$ & $1.70 \times 10^{4}$ \\
$[$Cr(acac)\sub{3}$]$ & $1.79 \times 10^{4}$ & $1.79 \times 10^{4}$ \\
$[$Cr(en)\sub{3}$]$\super{3+} & $2.06 \times 10^{4}$ & $2.16 \times 10^{4}$ \\
\hline
\end{tabular}
\end{center}

Based on the measured $\Delta$\sub{o}, the following spectrochemical series of ligands is produced:

\begin{center}
en $>$ acac $>$ H\sub{2}O $>$ Cl\super{-}
\end{center}

This is mostly consistent with the spectrochemical series provided in the lab procedure.\cite{lab} The only discrepancy is for the ligand acac (acetylacetone). As a ligand, it gains an extra proton: Hacac. One might expect its behavior to be similar to OH\super{-} or C\sub{2}O\sub{4}\super{2-}; however, it is quite clear from these experimental results that acac has some other interactions. Either way, these values for acac and water are very close. \\

One way to consider this discrepancy is to imagine where an extra proton would go on acac (on the other oxygen to form a second hydroxyl group), which is slightly more stable than the H\sub{3}O\super{+} ion, which means that acac is a better Br{\o}nsted acid. Thus, the inverse must be true that water is the better Br{\o}nsted base and has a lower $\Delta$\sub{o}. \\

For the ligand (H\sub{2}O)\sub{4}Cl\sub{2} on chromium(III), it is clear that water is replacing the halides (Cl\super{-}) over time. After six hours, the value of $\Delta$\sub{o} is the same as the ligand (H\sub{2}O)\sub{6}. Most of the change occurred within the first hour, which implies that the ligand is not stable and will interact in a solution of water to replace the Cl\super{-} ligands.


% answer leftover questions from the handout
\section{Questions\cite{lab}}
\begin{enumerate}[A.]
  \begin{item}
    Weak bands, with extinction coefficients $\sim$1\% of normal are observed at long wavelength for many Cr(III) complexes. What transitions do these bands correspond to? Why are these bands so weak? \\

    Transitions at long wavelengths are going to be ones lower in energy. It is likely that these are the less colorful absorptions (out of a human eye's range) that exist in the d $\rightarrow$ d transitions. These would also exhibit less absorbance because the selection rules regarding these transitions reduce their likelihood greatly. \\
  \end{item}

  \begin{item}
    Mn(II) and Fe(III) are examples of transition metal ions that are usually much more weakly colored than normal ``normal'' transition metal ions. Why are they so weakly colored? \\

    Both Mn\super{2+} and Fe\super{3+} have electron configuration of d\super{5}. This means that the d-orbitals are half-full. Normally, the charge-transfer transitions are what cause color in metal complexes. This can occur for metals with few electrons in the d-orbitals and the ligand sharing its electrons into the metal's d-orbitals (LMCT), or else this can be that the electrons in the d-orbitals of the metal may be shared with the ligands (MLCT). Since the d\super{5} configuration is exactly between both of these types of transfers, these ions will experience the least intensity of any chemical shifts. Thus, solutions of these complexes will be weakly colored. \\
  \end{item}

  \begin{item}
    Unlike most manganese(II) complexes, [Mn(CN)\sub{6}]\super{4-} is highly colored. Why? (\textit{Hint:} The ligand is not colored and has a large $\Delta$\sub{o} splitting.) \\

    The ligand CN\super{-} is a strong $\pi$-acceptor with a high $\Delta$\sub{o}, which means that it will cause the Mn\super{2+} ion complex to have a low-spin d\super{5} configuration. Because of the ligands, the manganese ion will share its d-orbital electrons to the ligands (an MLCT) and thus produce absorptions at a wavelength strongly enough that they produce a vibrant color.
  \end{item}
\end{enumerate}


% add the references
\raggedright
\bibliographystyle{achemso}
\bibliography{sources}


% and that's a wrap!
\end{document}
