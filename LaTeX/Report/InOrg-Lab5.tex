% by Sean McKenna

% This is a LaTeX version of a laboratory report
\documentclass[11pt]{article}

% necessary imports
\usepackage{enumerate}
\usepackage{amsmath}
\usepackage{amssymb}
\usepackage{makeidx}
\usepackage{graphicx}
\usepackage{hyperref}
\usepackage[utf8]{inputenc}
\usepackage{geometry}
\usepackage{booktabs}
\usepackage{array}
\usepackage{verbatim}
\usepackage{subfig}
\usepackage[super]{natbib}
\usepackage{mciteplus}

% page adjustments
\setlength{\parindent}{0pt}
\geometry{a4paper}

% expand margins?
%\addtolength{\oddsidemargin}{-.5in}
%\addtolength{\evensidemargin}{-.5in}
%\addtolength{\textwidth}{1in}
%\addtolength{\topmargin}{-.5in}
%\addtolength{\textheight}{1in}

% double-spaced?
%\linespread{2}

%define commands for super and sub scripts in text
\newcommand{\super}[1]{\ensuremath{^{\textrm{#1}}}}
\newcommand{\sub}[1]{\ensuremath{_{\textrm{#1}}}}

% title
\title{Kinetics of Reaction for Cr\super{3+} with EDTA}
\author{Sean McKenna}
\date{\today}

% begin the document
\begin{document}
\maketitle

% header info
\begin{center}
\begin{tabular}{lr}
Date Performed: March 20, 2012 & Lab Partner: Delaney Krisel \\
Instructor: Professor Cindy Strong
\end{tabular}
\end{center}


% describe the experiment's method
\section{Method}
Following the provided lab procedure,\cite{lab} reactions of a chromium(III) complex with EDTA were utilized to determine the overall order of the reaction with respect to H\super{+}. \\

This lab was a joint effort by the class to get all UV-vis spectra over time. Delaney prepared both the stock EDTA solution and the stock Cr(NO\sub{3})\sub{3}$\cdot$9H\sub{2}O solution. Additionally, the following table sums up the differences in procedure for each pH solution.

\begin{center}
\begin{tabular}{|c|c|c|c|c|c|}
\hline
\textbf{pH} & \textbf{Partners} & \textbf{UV-Vis} & \textbf{$\lambda$\sub{max} (nm)} & \textbf{$\Delta t$ (min)} & \textbf{Experiment Date} \\
\hline
3.50 & Delaney \& Sean & Cary 50 Bio & 575 & 5 & March 20th \\
4.40 & Molly \& Elizabeth & Agilant 8453 & 575 & 5 & March 15th \\
5.01 & Laura \& Adam & Agilant 8453 & 575 & 5 & March 20th \\
5.74 & Jose & Cary 50 Bio & 575 & 10 & March 15th \\
\hline
\end{tabular}
\end{center}


% show the reaction/s
\section{Reactions}
\begin{center}
3[C\sub{10}H\sub{14}N\sub{2}Na\sub{2}O\sub{8}$\cdot$2H\sub{2}O] (aq) + 4[Cr(NO\sub{3})\sub{3}$\cdot$9H\sub{2}O] (aq) $\rightarrow$ \\ Cr\sub{4}(C\sub{10}H\sub{12}N\sub{2}O\sub{8})\sub{3} (aq) + 6HNO\sub{3} (aq) + 6NaNO\sub{3} (aq) + 42H\sub{2}O (l) \\
\textit{or} \\
3[disodium salt-EDTA] + 4[chromium(III) nitrate nonahydrate] $\rightarrow$ tris-EDTA-chromate(III) + 6[nitric acid] + 6[sodium nitrate] + 42H\sub{2}O
\end{center}


% show some calculations
%\section{Sample Calculations}


% show any final results from the lab, tabular form
\section{Results}

The absorbance over time for the sample at pH = 3.50 was measured, in Figure 1. The sample had a spectra at $\Delta t = \infty$, which is shown in Figure 2. Using the absorbance at this time for $\lambda$ = 575 nm, the relative change in the absorbance was calculated. For all four sets of data provided by the class, the first and second-order reactions provided decent linear fits. All four of these data sets may be found in Figure 3 and Figure 5, with their slopes (or apparent rate constants $k$') and $R$\super{2} values in the table below.

\begin{center}
\begin{tabular}{|c|l|c|c|}
\hline
\textbf{Order of Cr(III)} & \textbf{pH} & \textbf{$k$' (with min\super{-1})} & \textbf{$R$\super{2}} \\
\hline \hline
2\super{nd} & 3.50 & 0.000317 & 0.966 \\
2\super{nd} & 4.40 & 0.00519 & 0.999 \\
2\super{nd} & 5.01 & 0.00258 & 0.998 \\
2\super{nd} & 5.74 & 0.0363 & 0.896 \\
\hline
1\super{st} & 3.50 & 0.000125 & 0.968 \\
1\super{st} & 4.40 & 0.00155 & 0.997 \\
1\super{st} & 5.01 & 0.000866 & 0.993 \\
1\super{st} & 5.74 & 0.00329 & 0.732 \\
\hline
\end{tabular}
\end{center}

Lastly, the apparent rate constants (k') were plotted with the pH's to establish the order of the reaction. This data can be seen in Figures 4 and 6 with a linear fit for the slope.

\begin{center}
\begin{tabular}{|c|c|c|}
\hline
\textbf{Order of Cr(III)} & \textbf{pH's for fit} & \textbf{Order of H\super{+}} \\
\hline
2\super{nd} & 4.40 \& 5.01 & -0.496 \\
1\super{st} & 4.40 \& 5.01 & -0.416 \\
\hline
\end{tabular}
\end{center}


% discuss what was found, compare to literature, answer specified questions
\section{Discussion}

The reaction appeared to be second-order with respect to Cr(III) according to the data, but the lab procedure\cite{lab} stated that Cr(III) was first-order. The pH of 5.74 was outside of the recommended range,\cite{lab} and the 3.50 pH data did not react enough (at least 75\%)\cite{lab} during the three hours. At both these pH's, anomalies appeared in the data analysis, and both were disregarded for calculating the order of the reaction with respect to H\super{+}. \\

Both first and second order for Cr(III) had close regression values, so it is possible that the data measured was not resolved carefully enough to rule out the second-order curve. \\

Analysis for each resulted in an inverse correlation of the reaction with respect to proton concentration, as expected.\cite{lab} Interestingly, the second-order Cr(III) analysis was much closer to $-\frac{1}{2}$ than the first-order analysis. Nevertheless, since two pH values had to be thrown out from the data, two-value line-fitting yielded a high uncertainty. It can be reported that the order of the reaction with respect to H\super{+} is close to $-\frac{1}{2}$. \\


% answer leftover questions from the handout
%\section{Questions}


% add the references
\raggedright
\bibliographystyle{achemso}
\bibliography{sources}


% and that's a wrap!
\end{document}
