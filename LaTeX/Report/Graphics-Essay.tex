% by Sean McKenna

\documentclass[11pt]{article}
\usepackage[utf8]{inputenc}

%%% PAGE DIMENSIONS
\usepackage{geometry}
\geometry{a4paper}

\usepackage{booktabs}
\usepackage{array}
\usepackage{verbatim}
\usepackage{subfig}
\usepackage{amssymb,amsmath}
\usepackage{graphicx} 
\usepackage{fancyhdr}
\pagestyle{fancy}
\renewcommand{\headrulewidth}{0pt}
\lhead{}\chead{}\rhead{}
\lfoot{}\cfoot{\thepage}\rfoot{}
\addtolength{\oddsidemargin}{-.875in}
\addtolength{\evensidemargin}{-.875in}
\addtolength{\textwidth}{1.75in}
\addtolength{\topmargin}{-.875in}
\addtolength{\textheight}{1.75in}

\linespread{2}

\title{Applications in Computer Science: Minecraft}
\author{Sean McKenna}

\begin{document}

\begin{center}
  \textbf{Applications in Computer Graphics: Java \& Minecraft} \\
  by Sean McKenna
\end{center}

Java is a programming language that is not dead. In fact, it is currently being used in the development of a large-scale indie game that is taking the world by force: Minecraft. This is evident in their gaining \$350,000 per day in sales back in September of 2010, likely only rising since that point in time. Who created this phenomenon? Swedish game-maker Markus ``Notch'' Persson began the work in Java that sprouted into a full game. In the game, users can destroy and create blocks. The game itself is still in beta and is available on the web and cross-platform, one of the many perks of being written in Java and the JVM. \\

Minecraft is built off of the Lightweight Java Game Library (LWJGL). This is how it is based off of Java, and it is the reason why there are so many different mods out there for the game. The game is not, however, open-source. It is a proprietary indie game with an available server version and a login through the main Minecraft server to authenticate users. While hacked versions do exist out there, the game is relatively cheap while still in beta so don't just rip the programmer ``Notch'' out of the well-earned \$21.88. \\

What appeals to people about Minecraft is its deceptively simple look and graphics. Using 8-bit technology, a whole world is crafted that looks like it could have existed years ago back on the older consoles and computers. However, what makes this game different is how it generates and renders all this data. Unlike normal computer games that are fixed with how they generate scenery and render graphics, Minecraft is a sandbox game designed to serve every need and to be adaptable. This means that worlds are generated on-the-fly. The basic building block of the world is a block. Blocks can be destroyed, mined, and crafted into other things. From this one can fashion tools and farm the land and create fire and light. The basics of survival are fundamental in the game. \\

What is really interesting is how Minecraft stores all its data and renders it. There are many algorithms it follows to generate terrain such as hillsides and bodies of water and caves and forests and sand dunes (and the list goes on and on). These algorithms make the simple scenery become interesting. Not only that, Minecraft was made with the intent of burrowing into caves and digging into the depths of the earth, encountering lava and many types of ore that can be smelted. These blocks are also randomly generated based on the algorithms in the game. The sheer amount of data that is generated and kept in memory in Minecraft makes it a game that would not have been feasible several years ago on current hardware. \\

All these blocks are stored in the map files, which ``regions'' of blocks, or 32x32x128 cubic block chunks. And these are all created relative to a ``level.dat'' file that links together these pieces and explains all the other details about the characters and inventories of the map. This is what makes Minecraft so flexible; many modders have used these open tools and other Java applications to access the Minecraft map files in different ways, to modify maps or to just view them in different forms. This community is growing and builds strength behind the game because of its open concept. The Minecraft site explains how one can read their map format using a class in Java and libraries in C and Python. This is just one example of a popular game that is revolutionary and is taking the PC world by force thanks to the development of graphics in Java.\\


References:\\
http://texyt.com/minecraft+persson+notch+indie+game+success+00127 \\
http://www.pcgamer.com/2010/07/29/community-heroes-notch-for-minecraft/

\end{document}
